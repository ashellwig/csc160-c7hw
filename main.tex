%
% Module 4 Chapter 7 Homework
% CSC160-C00: Computer Science I (C++)
% Author: Ashton Hellwig
%


\documentclass[a4paper, 10pt]{article}


  % Packages
  \usepackage[utf8]{inputenc}       % Encoding
  \usepackage[english]{babel}       % Internationalization
  \usepackage{soul}                 % Highlighting
  \usepackage{hyperref}             % Links (internal and external)
  \usepackage{fancyhdr}             % Headers and footers
  \usepackage[dvipsnames]{xcolor}   % Text Colors
  \usepackage{listings}             % Code Snippets
  \usepackage{algorithm}            % For TOC support
  \usepackage{algorithmicx}         % Algorithmic notation support
  \usepackage{algpseudocode}        % Algorithmic notation environments
  \usepackage{enumitem}             % Ordered lists
  \usepackage{geometry}             % Page layout
  \usepackage{graphicx}             % Image support
  \usepackage[toc, page]{appendix}  % Appendix
  \usepackage{bookmark}
  \usepackage{adjustbox}
  \usepackage{csquotes}
  \usepackage{amsthm}
  \usepackage{array}
  \usepackage{makecell}
  \usepackage{amsmath}
  \usepackage{amssymb}
  \usepackage{relsize}
  \usepackage{multicol}
  \usepackage{etoolbox,refcount}
  \usepackage{parcolumns}

%   \UseRawInputEncoding

  % Tables
  \renewcommand\theadalign{bc}
  \renewcommand\theadfont{\bfseries}
  \renewcommand\theadgape{\Gape[4pt]}
  \renewcommand\cellgape{\Gape[4pt]}

  % Lists
  \newcounter{countitems}
  \newcounter{nextitemizecount}
  \newcommand{\setupcountitems}{%
    \stepcounter{nextitemizecount}%
    \setcounter{countitems}{0}%
    \preto\item{\stepcounter{countitems}}%
  }
  \makeatletter
  \newcommand{\computecountitems}{%
    \edef\@currentlabel{\number\c@countitems}%
    \label{countitems@\number\numexpr\value{nextitemizecount}-1\relax}%
  }
  \newcommand{\nextitemizecount}{%
    \getrefnumber{countitems@\number\c@nextitemizecount}%
  }
  \newcommand{\previtemizecount}{%
    \getrefnumber{countitems@\number\numexpr\value{nextitemizecount}-1\relax}%
  }
  \makeatother    
  \newenvironment{AutoMultiColItemize}{%
  \ifnumcomp{\nextitemizecount}{>}{3}{\begin{multicols}{2}}{}%
  \setupcountitems\begin{itemize}}%
  {\end{itemize}%
  \unskip\computecountitems\ifnumcomp{\previtemizecount}{>}{3}{\end{multicols}}{}}



  % Colors
  \newcommand{\commentstylecolor}{\color{Gray}}
  \newcommand{\keywordstylecolor}{\color{MidnightBlue}}
  \newcommand{\stringstylecolor}{\color{ForestGreen}}
  \newcommand{\questioninput}{\color{Red}}
  \newcommand{\answertcolor}{\color{Green}}
  \newcommand{\myanswer}{\answertcolor{\hl}}

  % Symbols
  \newcommand{\answerflow}{\rotatebox[origin=c]{180}{$\Lsh$}}
  \newcommand{\toanswer}{\mathlarger{\mathlarger{\answerflow}}\quad}

  % Math
  \newcommand{\highlight}[1]{%
    \colorbox{green!50}{$\displaystyle#1$}}

  % Image Directory
  \graphicspath{ {screenshots/} }


  % Hyperlink Setup
  \hypersetup{
    colorlinks = true,
    urlcolor = blue,
    linkcolor = blue
  }


  % Syntax-Highlighting for Code Snippets
  \lstset{
    backgroundcolor=\color{white},
    breaklines=true,%
    captionpos=b,%
    frame=tb,%
    tabsize=4,%
    % numbers=left,%
    showstringspaces=false,%
    commentstyle=\commentstylecolor,%
    keywordstyle=\keywordstylecolor,%
    stringstyle=\stringstylecolor%
  }
  \lstset{literate=
  {á}{{\'a}}1 {é}{{\'e}}1 {í}{{\'i}}1 {ó}{{\'o}}1 {ú}{{\'u}}1
  {Á}{{\'A}}1 {É}{{\'E}}1 {Í}{{\'I}}1 {Ó}{{\'O}}1 {Ú}{{\'U}}1
  {à}{{\`a}}1 {è}{{\`e}}1 {ì}{{\`i}}1 {ò}{{\`o}}1 {ù}{{\`u}}1
  {À}{{\`A}}1 {È}{{\'E}}1 {Ì}{{\`I}}1 {Ò}{{\`O}}1 {Ù}{{\`U}}1
  {ä}{{\"a}}1 {ë}{{\"e}}1 {ï}{{\"i}}1 {ö}{{\"o}}1 {ü}{{\"u}}1
  {Ä}{{\"A}}1 {Ë}{{\"E}}1 {Ï}{{\"I}}1 {Ö}{{\"O}}1 {Ü}{{\"U}}1
  {â}{{\^a}}1 {ê}{{\^e}}1 {î}{{\^i}}1 {ô}{{\^o}}1 {û}{{\^u}}1
  {Â}{{\^A}}1 {Ê}{{\^E}}1 {Î}{{\^I}}1 {Ô}{{\^O}}1 {Û}{{\^U}}1
  {œ}{{\oe}}1 {Œ}{{\OE}}1 {æ}{{\ae}}1 {Æ}{{\AE}}1 {ß}{{\ss}}1
  {ű}{{\H{u}}}1 {Ű}{{\H{U}}}1 {ő}{{\H{o}}}1 {Ő}{{\H{O}}}1
  {ç}{{\c c}}1 {Ç}{{\c C}}1 {ø}{{\o}}1 {å}{{\r a}}1 {Å}{{\r A}}1
  {€}{{\euro}}1 {£}{{\pounds}}1 {«}{{\guillemotleft}}1
  {»}{{\guillemotright}}1 {ñ}{{\~n}}1 {Ñ}{{\~N}}1 {¿}{{?`}}1
}
  \newenvironment{alltt}{\ttfamily}{\par}


  % Page Configuration
  %% Style
  \pagestyle{fancy}

  %% Layout
  \geometry{%
  a4paper,%
  top=2.5cm,%
  bottom=2.5cm,%
  left=2.5cm,%
  right=2.5cm%
  }
  \setlength{\headheight}{12pt}
  \setlength{\floatsep}{12pt}

  %% Title page
  \title{Module 4 Chapter 7 Homework}
  \author{Ashton Hellwig}
  \date\today
  \setcounter{tocdepth}{3}

  %% Subsequent pages
  \lhead{CSC160}
  \rhead{Computer Science I (C++)}
  \lfoot{M4C7HW}
  \rfoot{A. Hellwig}

 
  % Document Content
\begin{document}
  \maketitle
  \tableofcontents
  \lstlistoflistings
  \newpage

  % Question 1
  \section{Question 1}
    Write C++ statements that do the following (assume the previous steps have been completed in
      succession):
      \begin{enumerate}[label=\Alph*.]
          \item Define an enum type, birdType, with the values PEACOCK, SPARROW, CANARY, PARROT, PENGUIN,
            OSTRICH, EAGLE, CARDINAL, and HUMMINGBIRD.
          \item Declare a variable bird of the type birdType.
          \item Assign CANARY to the variable bird.
          \item Advance bird to the next value in the list.
          \item Decrement bird to the previous value in the list.
          \item Output the value of the variable bird. Input value in the variable bird
      \end{enumerate}
    % Question 1 Solution
    \subsection{Solution}
      \begin{lstlisting}[language=c++,caption={Question 1 Solution},numbers=left]
#include <iostream>

using namespace std;

// Step A (Lines 6-16)
enum birdType {
  PEACOCK,
  SPARROW,
  CANARY,
  PARROT,
  PENGUIN,
  OSTRICH,
  EAGLE,
  CARDINAL,
  HUMMINGBIRD
};

int main() {
  birdType bird; // Step B
  
  bird = birdType::CANARY; // Step C
  
  bird = static_cast<birdType>(bird + 1); // Step D
  
  bird = static_cast<birdType>(bird + 1); // Step E
  
  // Step F (Lines 28-?)
  char first, second; // Declare `char` variables for selection structure
  // Output enum type NAME (not value, which would be: cout << bird << endl;)
  switch (bird) {
    case PEACOCK:
      cout << "Peacock";
      break;
    case SPARROW:
      cout << "Sparrow";
      break;
    case CANARY:
      cout << "Canary";
      break;
    case PARROT:
      cout << "Parrot";
      break;
    case PENGUIN:
      cout << "Penguin";
      break;
    case OSTRICH:
      cout << "Ostrich";
      break;
    case EAGLE:
      cout << "Eagle";
      break;
    case CARDINAL:
      cout << "Cardinal";
      break;
    case HUMMINGBIRD:
      cout << "Hummingbird";
      break;
    
  }
    
  return 0;
}

      \end{lstlisting}
    

  \newpage
  % Question 2
  \section{Question 2}
    Consider the following declaration:
        \begin{lstlisting}[language=c++]
enum fruitType{
    ORANGE,
    APPLE,
    BANANA,
    GRAPE,
    STRAWBERRY,
    MANGO,
    GUAVA,
    PINEAPPLE,
    KIWI
}; 
fruitType fruit;
        \end{lstlisting}
        \begin{enumerate}[label=\Alph*.]
            \item What is the value of \lstinline[language=c++]{static_cast<int>(STRAWBERRY)}?
            \item What is the value, if any, of the following expression? \\
                \lstinline[language=c++]{static_cast<fruitType>(static_cast<int>(MANGO) - 2)}
            \item What is the value, if any, of the following expression? \\
                \lstinline[language=c++]{static_cast<fruitType>(static_cast<int>(GRAPE) + 2)}
            \item What is the value, if any, of the expression: \\
                \lstinline[language=c++]{BANANA <= KIWI}
            \item What is the output, if any, of the following code?
                \begin{lstlisting}
for (fruit = BANANA; fruit < PINEAPPLE; fruit++)
    cout << static_cast<int>(fruit) << ", "; 

cout << endl;
                \end{lstlisting}
        \end{enumerate}
    % Question 2 Solution
    \subsection{Solution}
      \begin{enumerate}[label=\alph*.]
          \item \lstinline[language=c++]{static_cast<int>(STRAWBERRY)} = $4$
          \item \lstinline[language=c++]{static_cast<fruitType>(static_cast<int>(MANGO) - 2)}
            = $3$
          \item \lstinline[language=c++]{static_cast<fruitType>(static_cast<int>(GRAPE) + 2)}
            = $5$
          \item \lstinline[language=c++]{BANANA <= KIWI} = 1 (\lstinline[language=c++]{true})
          \item \lstinline[language=bash,columns=fixed]{Cannot increment expression of enum type fruitType}
      \end{enumerate}

  \newpage
  % Question 3
  \section{Question 3}
    Define an enumeration type triangleType with values EQUILATERAL, RIGHT, ISOSCELES, and SCALENE.
        Also, declare the variable triangle of type triangleType while defining this type.
    % Question 3 Solution
    \subsection{Solution}
      \begin{lstlisting}[language=c++,caption=Question 3 Solution,numbers=left]
enum triangleType{
  EQUILATERAL,
  RIGHT,
  ISOSCELES,
  SCALENE
} triangle;
      \end{lstlisting}

%   \newpage
  % Question 4
  \section{Question 4}
    What is wrong with the following program?
    \begin{lstlisting}[language=c++,numbers=left]
#include <iostream>

namespace mySpace
{
    const double RATE = 15.35;
    int a;
}

using namespace std;

int main()
{
    int b;
    cin >> b;
    a=b;
    cout << RATE << " " << a + 2 << " " << b
         << endl;
    Return 0;
}
    \end{lstlisting}
    % Question 4 Solution
    \subsection{Solution}
      \begin{center}
          \textbf{NB}: This solution is based on the line numbers \textit{above}, not in the question on CCCO.
      \end{center}
      
      \begin{enumerate}
          \item \hl{\textbf{Line \#15}}: \texttt{a} is undeclared in this scope. To use properly,
            the expression should have been using the scope-resolution operator (::) to access the
            variable within another namespace. Or, in addition to the
            \lstinline[language=c++,columns=fixed]{using namespace std;} statement,
            one would use \lstinline[language=c++,columns=fixed]{using namespace mySpace;}
            or even \lstinline[language=c++,columns=fixed]{using mySpace::a;} in order to forgo
            the scope-resolution operator in the \lstinline[language=c++,columns=fixed]{main()}
            function.
          \item \hl{\textbf{Line \#16}}: As in \texttt{line 15}, \texttt{RATE} is undefined here.
            In order to use the variable RATE from mySpace, the scope-resolution operator or a
            \lstinline[language=c++,columns=fixed]{using} statement needs to be implemented.
            The same goes for variable \texttt{a} in this line.
          \item \hl{\textbf{Line \#18}}: \lstinline[language=c++,columns=fixed]{return} is
            capitalized here. It should never be capitalized, but rather should be lowercase.
      \end{enumerate}


  \newpage
  % Question 5
  \section{Question 5}
    Suppose you have the following statements:
    \begin{lstlisting}[language=c++]
string str1, str2;

cin >> str1 >> str2; 
if (str1 == str2) 
cout << str1 + '!' << endl; 
else if (str1 > str2) 
cout << str1 + " > " + str2 << endl; 
else cout << str1 + " < " + str2 << endl;
    \end{lstlisting}
    
    Answer the following questions:
      \begin{enumerate}[label=\Alph*.]
          \item What is the output if the input is Programming Project?
          \item What is the output if the input is Summer Trip?
          \item What is the output if the input is Winter Cold?
      \end{enumerate}
    % Question 5 Solution
    \subsection{Solution}
      \begin{enumerate}[label=\alph*.]
          \item \lstinline[language=bash,columns=fixed]{Programming < Project}
          \item \lstinline[language=bash,columns=fixed]{Summer < Trip}
          \item \lstinline[language=bash,columns=fixed]{Winter > Cold}
      \end{enumerate}

%   \newpage
%   \appendix
%   \section{Question }
\end{document}